% Options for packages loaded elsewhere
\PassOptionsToPackage{unicode}{hyperref}
\PassOptionsToPackage{hyphens}{url}
%
\documentclass[
]{article}
\usepackage{lmodern}
\usepackage{amssymb,amsmath}
\usepackage{ifxetex,ifluatex}
\ifnum 0\ifxetex 1\fi\ifluatex 1\fi=0 % if pdftex
  \usepackage[T1]{fontenc}
  \usepackage[utf8]{inputenc}
  \usepackage{textcomp} % provide euro and other symbols
\else % if luatex or xetex
  \usepackage{unicode-math}
  \defaultfontfeatures{Scale=MatchLowercase}
  \defaultfontfeatures[\rmfamily]{Ligatures=TeX,Scale=1}
\fi
% Use upquote if available, for straight quotes in verbatim environments
\IfFileExists{upquote.sty}{\usepackage{upquote}}{}
\IfFileExists{microtype.sty}{% use microtype if available
  \usepackage[]{microtype}
  \UseMicrotypeSet[protrusion]{basicmath} % disable protrusion for tt fonts
}{}
\makeatletter
\@ifundefined{KOMAClassName}{% if non-KOMA class
  \IfFileExists{parskip.sty}{%
    \usepackage{parskip}
  }{% else
    \setlength{\parindent}{0pt}
    \setlength{\parskip}{6pt plus 2pt minus 1pt}}
}{% if KOMA class
  \KOMAoptions{parskip=half}}
\makeatother
\usepackage{xcolor}
\IfFileExists{xurl.sty}{\usepackage{xurl}}{} % add URL line breaks if available
\IfFileExists{bookmark.sty}{\usepackage{bookmark}}{\usepackage{hyperref}}
\hypersetup{
  pdftitle={govhack: construction of token system using datasets},
  pdfauthor={Anne},
  hidelinks,
  pdfcreator={LaTeX via pandoc}}
\urlstyle{same} % disable monospaced font for URLs
\usepackage[margin=1in]{geometry}
\usepackage{color}
\usepackage{fancyvrb}
\newcommand{\VerbBar}{|}
\newcommand{\VERB}{\Verb[commandchars=\\\{\}]}
\DefineVerbatimEnvironment{Highlighting}{Verbatim}{commandchars=\\\{\}}
% Add ',fontsize=\small' for more characters per line
\usepackage{framed}
\definecolor{shadecolor}{RGB}{248,248,248}
\newenvironment{Shaded}{\begin{snugshade}}{\end{snugshade}}
\newcommand{\AlertTok}[1]{\textcolor[rgb]{0.94,0.16,0.16}{#1}}
\newcommand{\AnnotationTok}[1]{\textcolor[rgb]{0.56,0.35,0.01}{\textbf{\textit{#1}}}}
\newcommand{\AttributeTok}[1]{\textcolor[rgb]{0.77,0.63,0.00}{#1}}
\newcommand{\BaseNTok}[1]{\textcolor[rgb]{0.00,0.00,0.81}{#1}}
\newcommand{\BuiltInTok}[1]{#1}
\newcommand{\CharTok}[1]{\textcolor[rgb]{0.31,0.60,0.02}{#1}}
\newcommand{\CommentTok}[1]{\textcolor[rgb]{0.56,0.35,0.01}{\textit{#1}}}
\newcommand{\CommentVarTok}[1]{\textcolor[rgb]{0.56,0.35,0.01}{\textbf{\textit{#1}}}}
\newcommand{\ConstantTok}[1]{\textcolor[rgb]{0.00,0.00,0.00}{#1}}
\newcommand{\ControlFlowTok}[1]{\textcolor[rgb]{0.13,0.29,0.53}{\textbf{#1}}}
\newcommand{\DataTypeTok}[1]{\textcolor[rgb]{0.13,0.29,0.53}{#1}}
\newcommand{\DecValTok}[1]{\textcolor[rgb]{0.00,0.00,0.81}{#1}}
\newcommand{\DocumentationTok}[1]{\textcolor[rgb]{0.56,0.35,0.01}{\textbf{\textit{#1}}}}
\newcommand{\ErrorTok}[1]{\textcolor[rgb]{0.64,0.00,0.00}{\textbf{#1}}}
\newcommand{\ExtensionTok}[1]{#1}
\newcommand{\FloatTok}[1]{\textcolor[rgb]{0.00,0.00,0.81}{#1}}
\newcommand{\FunctionTok}[1]{\textcolor[rgb]{0.00,0.00,0.00}{#1}}
\newcommand{\ImportTok}[1]{#1}
\newcommand{\InformationTok}[1]{\textcolor[rgb]{0.56,0.35,0.01}{\textbf{\textit{#1}}}}
\newcommand{\KeywordTok}[1]{\textcolor[rgb]{0.13,0.29,0.53}{\textbf{#1}}}
\newcommand{\NormalTok}[1]{#1}
\newcommand{\OperatorTok}[1]{\textcolor[rgb]{0.81,0.36,0.00}{\textbf{#1}}}
\newcommand{\OtherTok}[1]{\textcolor[rgb]{0.56,0.35,0.01}{#1}}
\newcommand{\PreprocessorTok}[1]{\textcolor[rgb]{0.56,0.35,0.01}{\textit{#1}}}
\newcommand{\RegionMarkerTok}[1]{#1}
\newcommand{\SpecialCharTok}[1]{\textcolor[rgb]{0.00,0.00,0.00}{#1}}
\newcommand{\SpecialStringTok}[1]{\textcolor[rgb]{0.31,0.60,0.02}{#1}}
\newcommand{\StringTok}[1]{\textcolor[rgb]{0.31,0.60,0.02}{#1}}
\newcommand{\VariableTok}[1]{\textcolor[rgb]{0.00,0.00,0.00}{#1}}
\newcommand{\VerbatimStringTok}[1]{\textcolor[rgb]{0.31,0.60,0.02}{#1}}
\newcommand{\WarningTok}[1]{\textcolor[rgb]{0.56,0.35,0.01}{\textbf{\textit{#1}}}}
\usepackage{graphicx,grffile}
\makeatletter
\def\maxwidth{\ifdim\Gin@nat@width>\linewidth\linewidth\else\Gin@nat@width\fi}
\def\maxheight{\ifdim\Gin@nat@height>\textheight\textheight\else\Gin@nat@height\fi}
\makeatother
% Scale images if necessary, so that they will not overflow the page
% margins by default, and it is still possible to overwrite the defaults
% using explicit options in \includegraphics[width, height, ...]{}
\setkeys{Gin}{width=\maxwidth,height=\maxheight,keepaspectratio}
% Set default figure placement to htbp
\makeatletter
\def\fps@figure{htbp}
\makeatother
\setlength{\emergencystretch}{3em} % prevent overfull lines
\providecommand{\tightlist}{%
  \setlength{\itemsep}{0pt}\setlength{\parskip}{0pt}}
\setcounter{secnumdepth}{-\maxdimen} % remove section numbering

\title{govhack: construction of token system using datasets}
\author{Anne}
\date{21/08/2021}

\begin{document}
\maketitle

We created a token system for 2 areas: \texttt{Council} and
\texttt{Type\ of\ Public\ space}.

For the \texttt{Type\ of\ Public\ space} we used the data from the
engagement report on public spaces and for the \texttt{Council} we used
the time series data and national data on disaster relief.

\textbf{Creating a token system for the \textbf{Council} Areas}

In order to create the token system, we used time series data to measure
the gradient (using a linear regression) of positive or negative growth
of different council areas over the years. Here were our consideraitons.

\begin{itemize}
\item
  We prioritised Council who were ``suffering'' i.e experiencing the
  most negative growth - to do this we ranked each variable based on
  magnitude of decline or increase.
\item
  We added all the rank indices of the variables together to create a
  weighted score. The variables included:
  \texttt{"Classification",\ "Population\ Density\ per\ capita/km2",\ "Socio-Economic",\ "Unemployment\ Rate"\ ,"Number\ of\ Active\ Businesses\ in\ LGA",\ "Recreational\ \&\ Cultural\ Expenditure\ per\ capita",\ "Open\ Public\ Space",\ "Number\ of\ Public\ Swimming\ Pools",\ "Number\ of\ Public\ Halls",\ "Number\ of\ Public\ Libraries"},
  from the original time series data. We also included data from the
  disaster relief (national data set) to promote travel and tourism in
  the council areas which had suffered disaster (in hopes to boost the
  economy in those areas). We merged these two datasets by \texttt{lga}
  and \texttt{Council} in the datasets.
\item
  We ranked the weighted score and then applied tokens to sections of
  the rankings i.e.~top 20 would have 15 tokens allocated, top 20-50
  would get 20 tokens, top 50-80 would get 25 tokens (increased in 5
  token increments)
\end{itemize}

\textbf{Creating a token system for the \textbf{Public Space} Areas}

Using the engagement report, we looked at the areas that the public
engaged with during COVID, areas that can be improved and areas that the
public loved. Then we manipulated the data as such:

\begin{itemize}
\tightlist
\item
  We looked at the types of public spaces that need a boost (with a
  lower percentage in the report) and then added more tokens to places
  that have not had as much attention previously.
\end{itemize}

\hypertarget{importing-and-cleaning-the-datasets}{%
\paragraph{Importing and cleaning the
datasets}\label{importing-and-cleaning-the-datasets}}

\begin{itemize}
\tightlist
\item
  Imported the datsets
\item
  Chose the variables needed to calculate the score
\item
  Cleaned the data: deleted the unneccessary terms such as:
  \texttt{"\ Council",\ "The\ Council\ of\ the\ City\ of\ ",\ "The\ Council\ of\ the\ Shire\ of\ ",\ "The\ Council\ of\ the\ Municipality\ of\ ",\ "\ City",\ "\ Shire",\ "\ Municipal",\ "Council\ of\ the\ City\ of\ "}
\end{itemize}

\hypertarget{matching-councils-merging-the-nsw-time-series-dataset-with-the-national-disaster-relief-bushfires-datasets}{%
\paragraph{Matching Councils: Merging the NSW time series dataset with
the National disaster relief + bushfires
datasets}\label{matching-councils-merging-the-nsw-time-series-dataset-with-the-national-disaster-relief-bushfires-datasets}}

\begin{itemize}
\tightlist
\item
  Merge the NSW time series dataset for public spaces with the national
  disaster relief data set
\end{itemize}

\begin{Shaded}
\begin{Highlighting}[]
\CommentTok{#bushfires_2019_2020 <- bushfires_2019_2020%>%mutate(across(where(is.character), as.factor))}
\CommentTok{#summary(bushfires_2019_2020)}
\CommentTok{#colnames(bushfires_2019_2020)}
\NormalTok{bushfires_}\DecValTok{2019}\NormalTok{_}\DecValTok{2020}\OperatorTok{$}\NormalTok{location_name<-}\StringTok{ }\KeywordTok{gsub}\NormalTok{(}\StringTok{"}\CharTok{\textbackslash{}\textbackslash{}}\StringTok{s*}\CharTok{\textbackslash{}\textbackslash{}}\StringTok{([^}\CharTok{\textbackslash{}\textbackslash{}}\StringTok{)]+}\CharTok{\textbackslash{}\textbackslash{}}\StringTok{)"}\NormalTok{,}\StringTok{""}\NormalTok{,}\KeywordTok{as.character}\NormalTok{(bushfires_}\DecValTok{2019}\NormalTok{_}\DecValTok{2020}\OperatorTok{$}\NormalTok{location_name))}


\CommentTok{#disaster_2019_2020 <- disaster_2019_2020%>%mutate(across(where(is.character), as.factor))}
\CommentTok{#summary(disaster_2019_2020)}
\NormalTok{disaster_}\DecValTok{2019}\NormalTok{_}\DecValTok{2020}\OperatorTok{$}\NormalTok{lga<-}\StringTok{ }\KeywordTok{removeWords}\NormalTok{(}\KeywordTok{as.character}\NormalTok{(disaster_}\DecValTok{2019}\NormalTok{_}\DecValTok{2020}\OperatorTok{$}\NormalTok{lga),}\StringTok{"Council"}\NormalTok{)}
\CommentTok{#summary(as.factor(disaster_2019_2020$lga))}


\CommentTok{#Match the councils in the disaster relief dataset with the councils in the time series dataset}
\NormalTok{index<-}\StringTok{ }\KeywordTok{which}\NormalTok{(}\KeywordTok{as.character}\NormalTok{(disaster_}\DecValTok{2019}\NormalTok{_}\DecValTok{2020}\OperatorTok{$}\NormalTok{lga) }\OperatorTok\StringTok{ }\KeywordTok{as.character}\NormalTok{(timeseries_data}\OperatorTok{$}\NormalTok{Council))}

\NormalTok{disaster_NSW<-}\StringTok{ }\NormalTok{disaster_}\DecValTok{2019}\NormalTok{_}\DecValTok{2020}\NormalTok{[index,] }\OperatorTok\StringTok{ }\NormalTok{dplyr}\OperatorTok{::}\KeywordTok{select}\NormalTok{(lga, cwlth_funding)}

\CommentTok{#Match the councils in the bushfire dataset with the councils in the time series dataset}
\NormalTok{index<-}\StringTok{ }\KeywordTok{which}\NormalTok{(}\KeywordTok{as.character}\NormalTok{(bushfires_}\DecValTok{2019}\NormalTok{_}\DecValTok{2020}\OperatorTok{$}\NormalTok{location_name) }\OperatorTok\StringTok{ }\KeywordTok{as.character}\NormalTok{(timeseries_data}\OperatorTok{$}\NormalTok{Council))}

\NormalTok{bushfire_NSW<-}\StringTok{ }\NormalTok{bushfires_}\DecValTok{2019}\NormalTok{_}\DecValTok{2020}\NormalTok{[index,]}

  
\NormalTok{bush_data <-}\StringTok{ }\NormalTok{bushfire_NSW }\OperatorTok\StringTok{  }
\StringTok{  }\KeywordTok{group_by}\NormalTok{(location_name) }\OperatorTok\StringTok{ }
\StringTok{  }\KeywordTok{summarise}\NormalTok{(}\DataTypeTok{n=}\KeywordTok{n}\NormalTok{())}
\end{Highlighting}
\end{Shaded}

\begin{verbatim}
## `summarise()` ungrouping output (override with `.groups` argument)
\end{verbatim}

\hypertarget{analysis-on-the-time-series-datasets-percentage-change}{%
\paragraph{Analysis on the time series datasets: percentage
change}\label{analysis-on-the-time-series-datasets-percentage-change}}

\begin{Shaded}
\begin{Highlighting}[]
\NormalTok{timeseries_data <-}\StringTok{ }\NormalTok{timeseries_data }\OperatorTok\StringTok{ }
\StringTok{  }\KeywordTok{na.omit}\NormalTok{() }
\KeywordTok{summary}\NormalTok{(timeseries_data}\OperatorTok{$}\StringTok{`}\DataTypeTok{Socio-Economic}\StringTok{`}\NormalTok{)}
\end{Highlighting}
\end{Shaded}

\begin{verbatim}
##    Min. 1st Qu.  Median    Mean 3rd Qu.    Max. 
##    1.00   35.00   70.00   70.53  105.00  153.00
\end{verbatim}

\begin{Shaded}
\begin{Highlighting}[]
\KeywordTok{summary}\NormalTok{(timeseries_data}\OperatorTok{$}\NormalTok{year)}
\end{Highlighting}
\end{Shaded}

\begin{verbatim}
##    Min. 1st Qu.  Median    Mean 3rd Qu.    Max. 
##    2014    2014    2015    2016    2018    2018
\end{verbatim}

\begin{Shaded}
\begin{Highlighting}[]
\KeywordTok{summary}\NormalTok{(timeseries_data}\OperatorTok{$}\NormalTok{year)}
\end{Highlighting}
\end{Shaded}

\begin{verbatim}
##    Min. 1st Qu.  Median    Mean 3rd Qu.    Max. 
##    2014    2014    2015    2016    2018    2018
\end{verbatim}

\begin{Shaded}
\begin{Highlighting}[]
\CommentTok{#getting the percentage change}
\CommentTok{# load scales to format dates on x-axis}

\CommentTok{#(timeseries_data$year,timeseries_data$`Socio-Economic` )}

\NormalTok{result_matrix <-}\StringTok{ }\KeywordTok{matrix}\NormalTok{(}\DataTypeTok{ncol=}\DecValTok{3}\NormalTok{)}
\ControlFlowTok{for}\NormalTok{ (i }\ControlFlowTok{in}\NormalTok{ timeseries_data}\OperatorTok{$}\NormalTok{Council)\{}
\NormalTok{  dat <-}\StringTok{ }\NormalTok{timeseries_data[}\KeywordTok{which}\NormalTok{(timeseries_data}\OperatorTok{$}\NormalTok{Council }\OperatorTok\StringTok{ }\NormalTok{i),]}
\NormalTok{  class_score <-}\StringTok{ }\KeywordTok{c}\NormalTok{(i, }\StringTok{"class_Score"}\NormalTok{, dat[,}\DecValTok{9}\NormalTok{])}
\NormalTok{  result_matrix <-}\StringTok{ }\KeywordTok{rbind}\NormalTok{(result_matrix, class_score)}

  \ControlFlowTok{for}\NormalTok{ (j }\ControlFlowTok{in} \KeywordTok{colnames}\NormalTok{(dat[,}\KeywordTok{c}\NormalTok{(}\DecValTok{3}\OperatorTok{:}\DecValTok{8}\NormalTok{)]))\{}
    \CommentTok{#print(j)}
\NormalTok{    indx <-}\StringTok{ }\KeywordTok{which}\NormalTok{(}\KeywordTok{colnames}\NormalTok{(dat) }\OperatorTok\StringTok{ }\NormalTok{j)}
    \CommentTok{#print(indx)}
\NormalTok{    mod <-}\StringTok{ }\KeywordTok{lm}\NormalTok{(dat[,indx]}\OperatorTok{~}\NormalTok{year, }\DataTypeTok{data =}\NormalTok{ dat)}
\NormalTok{    res <-}\StringTok{ }\KeywordTok{c}\NormalTok{(i, j, }\KeywordTok{coef}\NormalTok{(mod)[}\DecValTok{2}\NormalTok{])}
\NormalTok{    result_matrix <-}\StringTok{ }\KeywordTok{rbind}\NormalTok{(result_matrix, res)}
  
\NormalTok{  \}}
\NormalTok{\}}

\NormalTok{result_matrix <-}\StringTok{ }\KeywordTok{as.data.frame}\NormalTok{(result_matrix) }\OperatorTok\StringTok{ }\KeywordTok{na.omit}\NormalTok{()}
\NormalTok{result_matrix<-}\StringTok{ }\NormalTok{result_matrix }\OperatorTok\StringTok{ }
\StringTok{  }\KeywordTok{group_by}\NormalTok{(V1) }\OperatorTok\StringTok{ }
\StringTok{  }\KeywordTok{filter}\NormalTok{(}\KeywordTok{n}\NormalTok{()}\OperatorTok{>}\DecValTok{1}\NormalTok{) }\OperatorTok\StringTok{ }
\StringTok{  }\KeywordTok{as.data.frame}\NormalTok{() }\OperatorTok\StringTok{ }
\StringTok{  }\KeywordTok{mutate}\NormalTok{(}\DataTypeTok{year =} \KeywordTok{as.numeric}\NormalTok{(}\KeywordTok{as.character}\NormalTok{(year)))}\OperatorTok
\StringTok{  }\KeywordTok{mutate}\NormalTok{(}\DataTypeTok{V1 =}\NormalTok{ (}\KeywordTok{as.character}\NormalTok{(V1)))}\OperatorTok
\StringTok{  }\KeywordTok{mutate}\NormalTok{(}\DataTypeTok{year =} \KeywordTok{round}\NormalTok{(year,}\DecValTok{3}\NormalTok{))}\OperatorTok
\StringTok{  }\KeywordTok{rename}\NormalTok{(}\StringTok{"grad"}\NormalTok{ =}\StringTok{ "year"}\NormalTok{)}
\end{Highlighting}
\end{Shaded}

\hypertarget{calculating-the-score-and-token-system}{%
\paragraph{Calculating the score and token
system}\label{calculating-the-score-and-token-system}}

\begin{Shaded}
\begin{Highlighting}[]
\CommentTok{#### Calculating the score and token system}

\NormalTok{result_data<-}\StringTok{ }\KeywordTok{reshape}\NormalTok{(result_matrix, }\DataTypeTok{idvar=}\StringTok{"V1"}\NormalTok{, }\DataTypeTok{timevar=}\StringTok{"V2"}\NormalTok{, }\DataTypeTok{v.names=}\StringTok{"grad"}\NormalTok{, }\DataTypeTok{direction=}\StringTok{"wide"}\NormalTok{)}

\NormalTok{result_data<-}\StringTok{ }\KeywordTok{merge}\NormalTok{(result_data, bush_data, }\DataTypeTok{by.x =}\StringTok{"V1"}\NormalTok{, }\DataTypeTok{by.y =} \StringTok{"location_name"}\NormalTok{, }\DataTypeTok{all.x =} \StringTok{"TRUE"}\NormalTok{)}
\NormalTok{result_data <-}\StringTok{ }\NormalTok{result_data }\OperatorTok\StringTok{ }
\StringTok{  }\KeywordTok{mutate}\NormalTok{(}\DataTypeTok{bush =} \KeywordTok{ifelse}\NormalTok{(}\OperatorTok{!}\KeywordTok{is.na}\NormalTok{(n), }\DecValTok{0}\NormalTok{, }\DecValTok{20}\NormalTok{))}

\CommentTok{#merge(result_data, disaster_NSW, by.x = "V1", by.y = "lga", all.x = "TRUE")}

\NormalTok{result<-}\StringTok{ }\KeywordTok{cbind}\NormalTok{(}\DataTypeTok{council =} \KeywordTok{as.character}\NormalTok{(result_data}\OperatorTok{$}\NormalTok{V1),}
      \DataTypeTok{classification =} \KeywordTok{as.numeric}\NormalTok{(result_data}\OperatorTok{$}\NormalTok{grad.class_Score),}
      \DataTypeTok{ord_pop_dens =} \KeywordTok{as.numeric}\NormalTok{(}\KeywordTok{rank}\NormalTok{(result_data}\OperatorTok{$}\StringTok{`}\DataTypeTok{grad.Population Density per capita/km2}\StringTok{`}\NormalTok{)),}
      \DataTypeTok{ord_socio =} \KeywordTok{as.numeric}\NormalTok{(}\KeywordTok{rank}\NormalTok{(result_data}\OperatorTok{$}\StringTok{`}\DataTypeTok{grad.Socio-Economic}\StringTok{`}\NormalTok{)),}
      \DataTypeTok{ord_unemploy =} \KeywordTok{as.numeric}\NormalTok{(}\KeywordTok{rank}\NormalTok{(result_data}\OperatorTok{$}\StringTok{`}\DataTypeTok{grad.Unemployment Rate}\StringTok{`}\NormalTok{)),}
      \DataTypeTok{ord_activbus =} \KeywordTok{as.numeric}\NormalTok{(}\KeywordTok{rank}\NormalTok{(result_data}\OperatorTok{$}\StringTok{`}\DataTypeTok{grad.Number of Active Businesses in LGA}\StringTok{`}\NormalTok{)),}
      \DataTypeTok{ord_recexpcap =} \KeywordTok{as.numeric}\NormalTok{(}\KeywordTok{rank}\NormalTok{(result_data}\OperatorTok{$}\StringTok{`}\DataTypeTok{grad.Recreational & Cultural Expenditure per capita}\StringTok{`}\NormalTok{)),}
      \DataTypeTok{ord_pubspac =} \KeywordTok{as.numeric}\NormalTok{(}\KeywordTok{rank}\NormalTok{(result_data}\OperatorTok{$}\NormalTok{grad.PublicSpaces)),}
      \DataTypeTok{bush =}\NormalTok{ result_data}\OperatorTok{$}\NormalTok{bush) }\OperatorTok\KeywordTok{as.data.frame}\NormalTok{()}

\NormalTok{result[,}\KeywordTok{c}\NormalTok{(}\DecValTok{2}\OperatorTok{:}\DecValTok{9}\NormalTok{)]<-}\StringTok{ }\NormalTok{result[,}\KeywordTok{c}\NormalTok{(}\DecValTok{2}\OperatorTok{:}\DecValTok{9}\NormalTok{)] }\OperatorTok\StringTok{ }
\StringTok{  }\KeywordTok{mutate_if}\NormalTok{(is.factor, as.character)}\OperatorTok\StringTok{ }
\StringTok{  }\KeywordTok{mutate_if}\NormalTok{(is.character, as.numeric)}

\KeywordTok{str}\NormalTok{(result)}
\end{Highlighting}
\end{Shaded}

\begin{verbatim}
## 'data.frame':    129 obs. of  9 variables:
##  $ council       : Factor w/ 129 levels "Albury","Armidale Regional",..: 1 2 3 4 5 6 7 8 9 10 ...
##  $ classification: num  30 30 30 20 30 30 10 10 50 10 ...
##  $ ord_pop_dens  : num  100 61 97 48 78 73.5 36 68 116 40.5 ...
##  $ ord_socio     : num  13 102 63 81.5 26.5 50 128 57 118 46 ...
##  $ ord_unemploy  : num  31 13 25 62.5 76.5 120 124 34 84.5 74.5 ...
##  $ ord_activbus  : num  82 65 84 32 74 68 45 4 125 16 ...
##  $ ord_recexpcap : num  58 2 45 4 78 89 60 15 51 101 ...
##  $ ord_pubspac   : num  103 68.5 104.5 68.5 17 ...
##  $ bush          : num  20 0 0 20 20 0 0 20 20 20 ...
\end{verbatim}

\begin{Shaded}
\begin{Highlighting}[]
\NormalTok{result}\OperatorTok{$}\NormalTok{score <-}\StringTok{ }\OtherTok{NA}
\ControlFlowTok{for}\NormalTok{ (i }\ControlFlowTok{in} \DecValTok{1}\OperatorTok{:}\KeywordTok{nrow}\NormalTok{(result))\{}
\NormalTok{  result[i,]}\OperatorTok{$}\NormalTok{score <-}\StringTok{ }\NormalTok{result[i,}\DecValTok{2}\NormalTok{]}\OperatorTok{+}\NormalTok{result[i,}\DecValTok{3}\NormalTok{]}\OperatorTok{+}\NormalTok{result[i,}\DecValTok{4}\NormalTok{]}\OperatorTok{+}\NormalTok{result[i,}\DecValTok{5}\NormalTok{]}\OperatorTok{+}\NormalTok{result[i,}\DecValTok{6}\NormalTok{]}\OperatorTok{+}\NormalTok{result[i,}\DecValTok{7}\NormalTok{]}\OperatorTok{+}\NormalTok{result[i,}\DecValTok{8}\NormalTok{]}\OperatorTok{+}\NormalTok{result[i,}\DecValTok{9}\NormalTok{]}
\NormalTok{  result}\OperatorTok{$}\NormalTok{order_score <-}\StringTok{ }\KeywordTok{rank}\NormalTok{(result}\OperatorTok{$}\NormalTok{score)}
\NormalTok{\}}

\NormalTok{result<-result }\OperatorTok\StringTok{ }\KeywordTok{arrange}\NormalTok{(}\KeywordTok{desc}\NormalTok{(order_score))}
\NormalTok{result}\OperatorTok{$}\NormalTok{tokens_council <-}\StringTok{ }\KeywordTok{c}\NormalTok{(}\KeywordTok{rep}\NormalTok{(}\DecValTok{15}\NormalTok{,}\DecValTok{38}\NormalTok{), }\KeywordTok{rep}\NormalTok{(}\DecValTok{20}\NormalTok{,}\DecValTok{35}\NormalTok{), }\KeywordTok{rep}\NormalTok{(}\DecValTok{25}\NormalTok{, }\DecValTok{35}\NormalTok{), }\KeywordTok{rep}\NormalTok{(}\DecValTok{30}\NormalTok{,}\DecValTok{21}\NormalTok{))}

\KeywordTok{write.csv}\NormalTok{(result, }\DataTypeTok{file =} \StringTok{"tokens.csv"}\NormalTok{)}
\end{Highlighting}
\end{Shaded}

\hypertarget{adding-token-system-to-covid-data-from-the-engagment-report}{%
\paragraph{Adding token system to COVID data from the engagment
report}\label{adding-token-system-to-covid-data-from-the-engagment-report}}

\begin{Shaded}
\begin{Highlighting}[]
\NormalTok{covid_data}\OperatorTok{$}\NormalTok{cat <-}\StringTok{ "COVID"}
\NormalTok{space_improved_data}\OperatorTok{$}\NormalTok{cat <-}\StringTok{ "improve"}
\NormalTok{space_love_data}\OperatorTok{$}\NormalTok{cat <-}\StringTok{ "love"}

\NormalTok{extra_tokens <-}\StringTok{ }\KeywordTok{rbind}\NormalTok{(covid_data,}
\NormalTok{                      space_improved_data,}
\NormalTok{                      space_love_data) }

\NormalTok{extra_tokens <-}\StringTok{ }\NormalTok{extra_tokens}\OperatorTok\StringTok{ }
\StringTok{  }\KeywordTok{mutate}\NormalTok{(}\DataTypeTok{percent =} \KeywordTok{as.numeric}\NormalTok{(}\KeywordTok{as.character}\NormalTok{(percent)))}\OperatorTok
\StringTok{  }\KeywordTok{arrange}\NormalTok{(}\KeywordTok{desc}\NormalTok{(percent))}
\NormalTok{extra_tokens}\OperatorTok{$}\NormalTok{order_num <-}\StringTok{ }\KeywordTok{rank}\NormalTok{(extra_tokens}\OperatorTok{$}\NormalTok{percent)}
\NormalTok{extra_tokens}\OperatorTok{$}\NormalTok{extra_tokens <-}\StringTok{ }\KeywordTok{c}\NormalTok{(}\KeywordTok{rep}\NormalTok{(}\DecValTok{1}\NormalTok{,}\DecValTok{5}\NormalTok{), }\KeywordTok{rep}\NormalTok{(}\DecValTok{2}\NormalTok{, }\DecValTok{5}\NormalTok{), }\KeywordTok{rep}\NormalTok{(}\DecValTok{3}\NormalTok{, }\DecValTok{5}\NormalTok{), }\KeywordTok{rep}\NormalTok{(}\DecValTok{4}\NormalTok{,}\DecValTok{5}\NormalTok{), }\KeywordTok{rep}\NormalTok{(}\DecValTok{5}\NormalTok{,}\DecValTok{5}\NormalTok{), }\KeywordTok{rep}\NormalTok{(}\DecValTok{6}\NormalTok{,}\DecValTok{5}\NormalTok{), }\KeywordTok{rep}\NormalTok{(}\DecValTok{7}\NormalTok{,}\DecValTok{3}\NormalTok{))}

\KeywordTok{write.csv}\NormalTok{(extra_tokens, }\DataTypeTok{file =} \StringTok{"extra_tokens.csv"}\NormalTok{)}
\end{Highlighting}
\end{Shaded}

\end{document}
